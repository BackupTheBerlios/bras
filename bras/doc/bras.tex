\documentclass[12pt]{article}
\usepackage{a4}

\title{\textsf{bras}\\
another kind of `make'\\
\small \copyright 1996,1997 Harald Kirsch}
\author{\relax
Harald Kirsch\\
kir@iitb.fhg.de}

\newcommand{\bras}{\texttt{bras}}
\newcommand{\make}{\texttt{make}}
\newcommand{\brasfile}{\texttt{brasfile}}


%%%%%%%%%%%%%%%%%%%%%%%%%%%%%%%%%%%%%%%%%%%%%%%%%%%%%%%%%%%%%%%%%%%%%%%%
			   \begin{document}
%%%%%%%%%%%%%%%%%%%%%%%%%%%%%%%%%%%%%%%%%%%%%%%%%%%%%%%%%%%%%%%%%%%%%%%%
\maketitle
\tableofcontents

%%%%%%%%%%%%%%%%%%%%%%%%%%%%%%%%%%%%%%%%%%%%%%%%%%%%%%%%%%%%%%%%%%%%%%%%
\section{Why That?}
If you ever used \texttt{make}, you may have noticed that it has a few
odd features. Most annoying to me are
\begin{itemize}
\item the crazy syntax (tabs as a structuring element!),
\item the way variables are expanded,
\item the inability to handle dependencies in foreign directories,
\item the inability to distinguish between targets that merely need to
  exist and those that must be newer than their dependencies,
\item the lack of file-targets which always must be build,
\item the lack of control structures and 
\item the lack of decent pattern matching.
\end{itemize}

It is indisputable that \textbf{the} feature of \texttt{make}, ---
rule based command execution --- is most helpful in many situations;
in particular were it seems to be applied most often:
conditional compilation of large programs.

My idea was to combine the key features of \texttt{make} with all the
niceties of a command language, while improving the handling of
targets in foreign directories and the pattern matching
facilities. Since I personally like Tcl, not the least for its well
defined handling of command line substitution (see \texttt{Tcl(n)} and
no flames, please), I ended up with the present solution. A similar
solution might be possible with other command languages.

The following sections assume that you have a rough idea what \make\
is all about and what it does. Maybe I copy an introductory
section from a \make\ manual here the other day, but until then
\ldots.

%%%%%%%%%%%%%%%%%%%%%%%%%%%%%%%%%%%%%%%%%%%%%%%%%%%%%%%%%%%%%%%%%%%%%%%%
\section{Rule Commands}

Rule based command execution was the feature that made \make\ so
popular and distinguishes it from shells. And rules are what
\bras\ is all about. Without rules, \bras\ is identical to
\texttt{tclsh}. Three components constitute a \textit{rule}: a
\textit{target}, a list of \textit{dependencies} and a list of
\textit{commands}. The rule tells the rule-interpreter (i.e.\ a part
of \bras) that
\begin{itemize}
\item the commands must be executed to bring the target up-to-date
(whatever that means, see below), and that
\item the commands can only be executed, if all dependencies are
up-to-date.
\end{itemize}

What does \textit{up-to-date} mean? Historic \make\ has only one
notion of \textit{up-to-date}: the target is up-to-date, if it exists
and is newer than all its dependencies. Because this is to simple in
some situations, \bras\
has three types of rules implementing different
meanings of \textit{up-to-date}:

\begin{itemize}
\item The \texttt{Newer}-rule most closely resembles \make's notion of
\textit{up-to-date}.
\item The \texttt{Exist}-rule considers a target to be up-to-date,
if a file of that name exists. The files modification date as well as
any modification dates of dependencies are irrelevant.
\item The \texttt{Always}-rule considers the target to be never
up-to-date, i.e.\ the rule's commands are always executed if
\bras\ is asked to bring this target up-to-date.
\end{itemize}

Looking a bit closer at the meaning of \texttt{Newer}-rules (and
\make's rules),
dependencies serve two purposes. On the one hand, they are potential
inputs used by the commands and must be up-to-date before the commands
can be executed. On the other hand they trigger the rule if they are
newer than the target. I could barely resist to implement in \bras\ a
construct to make these two facets of dependencies explicit, but
finally decided that the distinction is mainly academic.

All that said, \bras's reasoning can be described easily.  After
reading in the brasfile, the target of the very first rule or the
targets given on the command line are considered. For each target it
is decided whether it is up-to-date or not, which is trivial for
\texttt{Always}-rules and easy for \texttt{Exist}-rules, but a bit
elaborate for \texttt{Newer}-rules. If the target is not up-to-date,
\bras\ considers all dependencies in turn and brings them
up-to-date. Then the rule's command is executed and whenever it
finishes without error, \bras\ assumes that the target is
up-to-date. In particular, \bras\ does not check whether a file with
the target's name was produced or not.

The general form of a rule is
\begin{quote}
\textit{Rulename} \textit{target} \textit{dependencies} \textit{command}
\end{quote}
where \textit{Rulename} is one of \texttt{Newer}, \texttt{Exist} or
\texttt{Always}, \textit{target} is one word, \textit{dependencies} is
a list of words and \textit{command} is (a list of) Tcl-command(s).

The \textit{dependencies} may be empty as well as \textit{command}. If
\textit{command} is empty, \bras\ substitutes a default command if it
can find one as described in section~\ref{secSuffixAsDefCmd}. The
command is considered to be empty, if its string length is zero. To
prevent \bras\ from looking for a default command, use \verb|{ }|,
i.e. a command containing some blanks.

The following sections describe the three types of rules in detail.

%%%%%%%%%%%%%%%%%%%%%%%%%%%%%%%%%%%%%%%%%%%%%%%%%%
\subsection{The \texttt{Newer}-rule}
\label{secNewer}
The most familiar rule for \make-users is probably the
\texttt{Newer}-rule. It has the form:
\begin{quote}
  \texttt{Newer} \textit{target} \textit{dependencies} \textit{command}
\end{quote}
An example is
\begin{verbatim}
  Newer yummy.o {yummy.c yummy.h} {
    cc $CFLAGS -o $target $deps
  }
\end{verbatim}
For the meaning of \texttt{target} and \texttt{deps} see
section~\ref{secCmdExec}. 
The \textit{command} is executed whenever at least one of the
following conditions holds:
\begin{itemize}
\item No file of name \textit{target} exists in the current directory.
\item A file with name \textit{target} exists but is older than an
existing file contained in the dependency list.
\item Considering each dependency in turn results in the decision to
rebuild one or more of them.
\end{itemize}

%%%%%%%%%%%%%%%%%%%%%%%%%%%%%%%%%%%%%%%%%%%%%%%%%%
\subsection{The \texttt{Exist}-rule}

One thing that often annoys me using \make\ is, that it is not
possible to write a decent rule to manage the existence of a
directory. Therefore I introduced the \texttt{Exist}-rule:
\begin{quote}
  \texttt{Exist} \textit{target} \textit{dependencies} \textit{command}
\end{quote}
An example is
\begin{verbatim}
  Exist /usr/local/bras-0.1 {} {
    mkdir -p $target
  }
\end{verbatim}
The \textit{command} is executed only if there is no file with name
\textit{target}. If \textit{dependencies} are given, they can never
trigger the rule. However if the rule triggers because \textit{target}
does not exist, then all \textit{dependencies} are brought up-to-date,
\textbf{before} the \textit{command} is executed.

As and example, the following rules make sure that
\texttt{/usr/local} is not accidently created by \texttt{mkdir -p} if
it does not yet exist.

\begin{verbatim}
  Exist /usr/local/bras-0.1 {/usr/local} {
    mkdir -p $target
  }
  Exist /usr/local {} {
    puts "Sorry, /usr/local must exist before I can proceed."
    exit 1
  }	
\end{verbatim}
The first rule has dependency \texttt{/usr/local}. If the target
\texttt{/usr/local/bras-0.1}
does not exist, the dependency \texttt{/usr/local} is considered
before execution of \texttt{mkdir}. The second rule checks whether it
exists. If this is the case, it does nothing and the first rule
proceeds with the \texttt{mkdir}-command. But if
\texttt{/usr/local} does not exist, the message is printed and \bras\
is terminated.

%%%%%%%%%%%%%%%%%%%%%%%%%%%%%%%%%%%%%%%%%%%%%%%%%%%%%%%%%%%%
\subsection{The \texttt{Always}-rule}

The \texttt{Always}-rule has the form:

\begin{quote}
  \texttt{Always} \textit{target} \textit{dependencies} \textit{command}
\end{quote}

An example is
\begin{verbatim}
  Always clean {@lib/clean @src/clean @doc/clean} {
    puts "Everbody's clean now"
  }
\end{verbatim}

The \textit{command} is executed whenever \texttt{clean} is
considered, independent of the existence of \textit{target} or the
modification dates of \textit{dependencies}.  But \textbf{before}
\textit{command} is executed, all
\textit{dependencies} are brought up-to-date.

%%%%%%%%%%%%%%%%%%%%%%%%%%%%%%%%%%%%%%%%%%%%%%%%%%%%%%%%%%%%%%%%%%%%%%%%
\section{Pattern Rules}

Pattern rules extend on the idea of \make's suffix (or implicit)
rules and serve two purposes:

\begin{enumerate}
\item If there is no rule for a target under consideration, \bras\
tries to find a matching pattern rule and creates a real rule from it on
the fly. The matching process is described in
section~\ref{secSuffixToReal}.
\item If a rule with an empty command is triggered, \bras\ tries
to find a matching pattern rule to substitute its command as a default.
\end{enumerate}

Every rule type has its corresponding pattern rule type. They
are called
\texttt{PatternNewer}, \texttt{PatternExist} and
\texttt{PatternAlways}.

The general form of pattern rules is
\begin{quote}
\texttt{Pattern}\textit{Type regexp depprocs command}
\end{quote}

where 

\begin{description}
\item[\textit{Type}] is either \texttt{Newer}, \texttt{Exist} or
\texttt{Always},
\item[\textit{regexp}] is a regular expression selecting the targets
for which this rule will be used,
\item[\textit{depproc}] is a list of Tcl-procedures that generate
dependencies from a given target and
\item[\textit{command}] is the command to be used in a derived real
rule or as a default command.
\end{description}

The details about \textit{regexp} and \textit{depprocs} are described
in sections~\ref{secSuffixToReal} and~\ref{secSuffixAsDefCmd}.

An example pattern rule to make an object file from a C source file is:
\begin{verbatim}
  SuffixNewer .*\.o .c {
    $CC -c $CFLAGS [file rootname $target].c
  }
\end{verbatim}

%%%%%%%%%%%%%%%%%%%%%%%%%%%%%%%%%%%%%%%%
\subsection{How pattern rules are used to create real rules}
\label{secSuffixToReal}

As mentioned above, pattern rules are used to create a new rule, if
there is no explicit rule available for the target under
consideration. 
For a given target, the pattern rule to use is selected as described
below.

In the order they were specified, all known pattern rules are checked
as to whether their regular expression matches the target. To make
sure that the whole word matches, regular expressions of pattern rules
get an implicit \verb+^+ and \texttt{\$} prefixed and suffixed. Put
another way: If \texttt{target} is the variable containing the target
under consideration and \texttt{rexp} is the regular expression of a
pattern rule, \bras\ executes an equivalent of
\begin{verbatim}
    regexp "^$rexp\$" $target
\end{verbatim}
to check whether the pattern rule is a \textit{candidate} or not.

Suppose a candidate pattern rule is found for a target named
\texttt{bla.o}. Then two more steps are necessary
to test whether the candidate can really be used.
The first step derives possible dependencies
from \texttt{bla} by calling
an equivalent of
\begin{quote}
    \texttt{Dep}$d$ \texttt{\$target}
\end{quote}
for every element $d$ of the \textit{depprocs}-list of the candidate
pattern rule. For example if \textit{depprocs} contains \texttt{.c}
and \texttt{.s}, \bras\ calls
\begin{verbatim}
    Dep.c $target
    Dep.s $target
\end{verbatim}
to derive possible dependency names. If the procedure \texttt{Dep$d$}
was not defined in a brasfile, the following
default is used:
\begin{quote}
    \verb|proc Dep|$d$\verb|{target} {|\\
    \texttt{\hspace*{2em}return [file root \$target]$d$}\\
    \verb|}|
\end{quote}
It simply strips the file extension from the target and suffixes the
result with $d$. More elaborate
procedures can be defined as needed.

The second step checks if one of
the derived dependencies denotes an existing file.
If the answer is yes, \bras\ creates a real rule with
\begin{itemize}
\item the target under consideration as the rule target,
\item the file found as the only dependency and
\item the command of the pattern rule as the rule's command,
\end{itemize}
and then proceeds as if there had never been a search for a pattern
rule.

%%%%%%%%%%%%%%%%%%%%%%%%%%%%%%%%%%%%%%%%
\subsection{How default commands are derived from pattern rules}
\label{secSuffixAsDefCmd}
Besides being rule patterns, pattern rules are also used to define
default commands, i.e. commands which are used in case a real rule was
defined with an empty command.

I would like to point out that the two uses of pattern rules are
totally unrelated, and that one could as well have introduced the
concept of default commands explicitly. While creation of a new rule
from a pattern rule happens while \bras\ is reasoning about which rules
must be triggered in which order, the extraction of default commands
happens later when the rules are actually triggered. 

Whenever \bras\ triggers a rule with an empty command, it tries to
find a pattern rule in much the same way as described in
section~\ref{secSuffixToReal}, except that in the last step it does
not try to find a file, but instead tests if one of the derived
dependencies is equal to one of the dependencies of the given real
rule. If the test succeeds, the command of the pattern rule is used as
the default command for the real rule.

Suppose the real rule with the empty command is
\begin{verbatim}
  Newer hack.o {hack.c hack.s} {}
\end{verbatim}
and that
\begin{verbatim}
  SuffixNewer .*\.o .s {
    cc -o $target $newer
  }
\end{verbatim}
is the candidate pattern rule. It is obviously good enough
because the default dependency procedure \texttt{Dep.s} returns
\texttt{hack.s} if called with argument \texttt{hack.o}.

To prevent \bras\ from looking for a default command, use a command
which is \textit{almost} empty, i.e. one which contains only blank
space. It is most convenient to write something like
\begin{verbatim}
  Newer hack.o {hack.c} {
  }
\end{verbatim}
to prevent \bras\ from looking for default commands.

%%%%%%%%%%%%%%%%%%%%%%%%%%%%%%%%%%%%%%%%%%%%%%%%%%%%%%%%%%%%%%%%%%%%%%%%
\section{Dependencies in Foreign Directories}

Most annoying to me using \make\ is its unwieldy handling of
dependencies in other directories. The usual solution is to call
\make\ recursively after changing to the foreign directory. But this
breaks the chain of reasoning, because there is no communication
between parent- and child-make as to whether any target was build or
whether all targets were up-to-date.

This is different with \bras. Whenever a dependency starts with
\texttt{@} and points to another directory, \bras\ expects
a \texttt{brasfile} in that directory which describes how the
dependency --- if considered as a target --- has to be constructed.
The following excerpt from a \texttt{brasfile} is an example:

\begin{verbatim}
  Always clean {@lib/clean @src/clean @doc/clean} {
    puts "Cleaning done."
  }
\end{verbatim}

The rule might typically be contained in a top-level
\texttt{brasfile}. Whenever 
\begin{verbatim}
    bras clean
\end{verbatim}
is called, the rule is considered, and consequently all
dependencies in turn are considered as a target. For example
\texttt{@lib/clean} tells \bras\ that there is a subdirectory
\texttt{lib} with a \texttt{brasfile} which
describes how to handle the target \texttt{clean}. To learn
the rule, \bras\ reads \texttt{lib/brasfile} and acts accordingly.
In particular \bras\ changes to directory \texttt{lib}
before executing any command associated with a rule in
\texttt{lib/brasfile}. The other targets are handled similarly.


%%%%%%%%%%%%%%%%%%%%%%%%%%%%%%%%%%%%%%%%%%%%%%%%%%%%%%%%%%%%%%%%%%%%%%%%
\section{How Commands are Executed}
\label{secCmdExec}

All commands derived from rules are executed on the global level
so that they have access only to all global variables. In addition
to the global variables defined in the
\texttt{brasfile}, \bras\ sets the variables \texttt{target},
\texttt{deps} and \texttt{newer} respectively to the rule's target, its
dependencies and the dependencies that are newer than the target. If
the target does not exist, all dependencies are considered newer and
\texttt{deps} and \texttt{newer} will contain the same value(s).

Non-internal commands are automatically executed
via \texttt{exec}. But please remember that \bras\ is basically
Tcl and unlike \texttt{sh}, \texttt{tclsh} and therefore \bras\ does no
globbing on the command line. Consequently, the rule
\begin{verbatim}
    Always clean {
      rm *.o *.a *~
    }
\end{verbatim}
does not work as expected. You have to resort to
\begin{verbatim}
   Exist clean {
      rm [glob -nocomplain -- *.o *.a *~]
    }
\end{verbatim}
    
A delicate problem is variable substitution in
external commands. Suppose \texttt{CFLAGS} is set to 
\texttt{"-g -Wall"} and there is a command like 

\begin{verbatim}
    cc $CFLAGS -c $target
\end{verbatim}

If external commands were only exec'ed you couldn't expect this to
work, because it were equivalent to the tcl-script

\begin{verbatim}
    exec cc $CFLAGS -c $deps
\end{verbatim}

which calls \texttt{cc} with just 3 arguments resulting effectively in

\begin{verbatim}
    cc "-g -Wall" -c whatever.c
\end{verbatim}

To make it work, \bras\ actually does

\begin{verbatim}
    eval exec cc $CFLAGS -c $target
\end{verbatim}

but beware of unwanted flattening in commands containing braces: Due
to \texttt{eval} one level of braces
disappears so that

\begin{verbatim}
    sed -e {s/^[/]*//} bla >bli
\end{verbatim}

after \texttt{eval} is actually called as

\begin{verbatim}
    exec -e s/^[/]*// bla >bli
\end{verbatim}

resulting in tcl complaining about the unknown command in
\texttt{[/]}. The remedy is an extra level of braces. While this might
seem clumsy, I find it still much more elegant than the backslashing
necessary in makefiles.



%%%%%%%%%%%%%%%%%%%%%%%%%%%%%%%%%%%%%%%%%%%%%%%%%%%%%%%%%%%%%%%%%%%%%%%%
\section{Use of automatically generated Dependencies}

Most C-compilers are able to generate makefile-dependencies for
source files by determining which files are included with
\texttt{\#include} directives. For example \texttt{gcc} can be
instructed to do so with option \texttt{-M} while SUN's Solaris
C-compiler uses \texttt{-xM}. Because this is a valuable feature,
\bras\ has the ability to read and understand a restricted type of
\make-dependencies. The command
\begin{quote}
  \texttt{sourceDeps} \textit{file}
\end{quote}
reads a dependency file created by the C-compiler and creates the
appropriate \texttt{Newer}-rules. The command prints a warning, if
the given file does not exist.
   
%%%%%%%%%%%%%%%%%%%%%%%%%%%%%%%%%%%%%%%%%%%%%%%%%%%%%%%%%%%%%%%%%%%%%%%%
\section{Limitations and Known Bugs}


\begin{enumerate}
\item
Since I did not yet test \bras\ with a really big project like say
\texttt{Xemacs} or \texttt{Tcl/Tk}, I don't know if the recursive
inclusion of many \texttt{brasfile}s will lead to performance
problems.

\item
When sourcing \texttt{brasfile}s from other directories, \bras\
currently does not construct a local scope for the variables used
there. The effect is that it is not possible to use the same variable
name, e.g.\ \texttt{SRC} and \texttt{OBJ}, in the brasfiles
of all subdirectories. This may be corrected in the future by using a
separate interpreter in every directory.

\end{enumerate}

%%%%%%%%%%%%%%%%%%%%%%%%%%%%%%%%%%%%%%%%%%%%%%%%%%%%%%%%%%%%%%%%%%%%%%%%
\section{The Name of the Game}

First I wanted to call it \texttt{brassel} which is the imperative of
\texttt{brasseln}, a verb you probably won't find in the
dictionary. You may need to ask someone from around K\"oln, but be
prepared to get the answer that (s)he knows what it means but cannot
explain it. I think the description "`working concentrated, busily but
without stress"' is quite close to the real meaning.

Well, since \texttt{brassel} is much to long for a good
Un*x utility, I shortened it to \bras.
    
\end{document}
